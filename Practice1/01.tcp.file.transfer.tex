\documentclass[11pt, a4paper]{article}

% --- UNIVERSAL PREAMBLE BLOCK ---
\usepackage[a4paper, top=2.5cm, bottom=2.5cm, left=2cm, right=2cm]{geometry}
\usepackage{fontspec}

\usepackage[english, bidi=basic, provide=*]{babel}

\babelprovide[import, onchar=ids fonts]{english}

% Set default/Latin font to Sans Serif in the main (rm) slot
\babelfont{rm}{Noto Sans}

\usepackage{graphicx} % For placeholder image/figures
\usepackage{caption}
\usepackage{subcaption}
\usepackage{booktabs} % For professional tables
\usepackage{listings} % For code listing
\usepackage{amsmath}  % For mathematical environments
\usepackage{tikz}     % For simple diagrams
\usetikzlibrary{shapes.geometric, arrows.meta, positioning}
\usepackage{hyperref} % Must be the last package loaded

% Define color for code highlighting
\definecolor{codegray}{rgb}{0.5,0.5,0.5}
\definecolor{codebackground}{rgb}{0.95,0.95,0.95}
\definecolor{codepurple}{rgb}{0.58,0,0.82}
\definecolor{codeblue}{rgb}{0.2,0.2,0.7}
\definecolor{codegreen}{rgb}{0,0.6,0}

% Global settings for the listings package
\lstset{
    language=C, % Default language
    basicstyle=\small\ttfamily,
    keywordstyle=\color{codeblue}\bfseries,
    identifierstyle=\color{black},
    commentstyle=\color{codegreen}\ttfamily,
    stringstyle=\color{codepurple},
    showstringspaces=false,
    numberstyle=\tiny\color{codegray},
    numbers=left,
    stepnumber=1,
    numbersep=5pt,
    backgroundcolor=\color{codebackground},
    tabsize=2,
    frame=single,
    frameround=tttt,
    breaklines=true,
    breakatwhitespace=true,
    captionpos=b,
    escapechar=\%,
    morekeywords={socket, bind, listen, accept, connect, send, recv, close}
}

\title{A Primer on TCP/IP Networking}
\author{The Networking Guide}
\date{\today}

\begin{document}

\maketitle

\begin{abstract}
This document serves as a brief introduction to the core concepts of the TCP/IP model, focusing on the two fundamental protocols: Transmission Control Protocol (TCP) and Internet Protocol (IP). We explore the layered structure of the model, detail the roles of TCP (connection-oriented, reliable) and IP (connectionless, best-effort), and provide a simple code example demonstrating a basic TCP socket connection flow.
\end{abstract}

\section{The TCP/IP Model (Internet Protocol Suite)}

The TCP/IP model is the foundational framework of the Internet. It is often described in comparison to the older, more complex OSI model. While the OSI model defines seven distinct layers, TCP/IP consolidates these functions into four or five practical layers. We will focus on the four-layer model:

\begin{enumerate}
    \item \textbf{Application Layer:} Provides network services for user applications (e.g., HTTP, FTP, SMTP).
    \item \textbf{Transport Layer:} Responsible for end-to-end communication services, primarily handled by TCP and UDP.
    \item \textbf{Internet Layer:} Responsible for logical addressing and routing across network boundaries (IP is the main protocol here).
    \item \textbf{Network Access Layer:} Handles the physical details of how data is sent across the physical link (e.g., Ethernet, Wi-Fi).
\end{enumerate}

\begin{figure}[htbp]
  \centering
  % Using a TikZ diagram to illustrate the layers
  \begin{tikzpicture}[
      node distance=1.5cm,
      layer/.style={rectangle, draw, fill=blue!20, minimum width=4cm, minimum height=1cm, align=center},
      arrow/.style={-latex, thick}
    ]

    % Define the nodes (Layers)
    \node (App) [layer] {Application Layer (HTTP, SMTP)};
    \node (Trans) [layer, below=of App] {Transport Layer (TCP, UDP)};
    \node (Internet) [layer, below=of Trans] {Internet Layer (IP)};
    \node (Link) [layer, below=of Internet] {Network Access Layer (Ethernet, Wi-Fi)};

    % Draw the vertical arrows/lines
    \draw[arrow] (App.south) -- (Trans.north);
    \draw[arrow] (Trans.south) -- (Internet.north);
    \draw[arrow] (Internet.south) -- (Link.north);

  \end{tikzpicture}
  \caption{The Four-Layer TCP/IP Model Structure. [Image of the Four-Layer TCP/IP Model Structure]}
  \label{fig:tcpip_model}
\end{figure}

\section{Internet Protocol (IP) - The Addressing Backbone}

The Internet Protocol (IP) operates at the Internet Layer and is responsible for addressing and routing data packets (datagrams) from a source host to a destination host across one or more networks.

\subsection{Key Characteristics of IP}
\begin{itemize}
    \item \textbf{Connectionless:} IP does not establish a connection before sending data. Each packet is treated independently.
    \item \textbf{Unreliable (Best-Effort):} IP makes no guarantees regarding delivery, ordering, or duplicate protection. Packets can be lost, corrupted, or arrive out of order.
    \item \textbf{IP Addressing:} Every device connected to the network must have a unique IP address (IPv4 or IPv6) to be locatable by routers.
\end{itemize}

\section{Transmission Control Protocol (TCP) - The Reliability Enforcer}

TCP operates at the Transport Layer and provides a reliable, ordered, and error-checked delivery stream between applications running on hosts. It is the primary protocol used by most network applications, including web browsing (HTTP), email (SMTP), and file transfer (FTP).

\subsection{Key Features of TCP}
\begin{itemize}
    \item \textbf{Connection-Oriented:} TCP establishes a logical connection, known as the \emph{three-way handshake}, before data transfer begins.
    \item \textbf{Reliable Delivery:} TCP uses sequence numbers, acknowledgements (ACKs), and retransmission mechanisms to ensure all data arrives correctly and in order.
    \item \textbf{Flow Control:} Prevents a fast sender from overwhelming a slow receiver.
    \item \textbf{Congestion Control:} Adjusts the rate of data transmission to prevent network congestion.
\end{itemize}

\subsection{The Three-Way Handshake}
The connection establishment process is fundamental to TCP's reliability. It consists of three steps:
\begin{enumerate}
    \item \textbf{SYN (Synchronize):} The client sends a SYN segment to the server.
    \item \textbf{SYN-ACK (Synchronize-Acknowledge):} The server responds with a SYN-ACK segment.
    \item \textbf{ACK (Acknowledge):} The client sends an ACK segment to the server, and the connection is established.
\end{enumerate}

\section{Socket Programming Example (C Language)}

The following C code illustrates the basic structure of a simple TCP server. The server uses the `socket`, `bind`, `listen`, and `accept` calls to handle incoming connections.

\begin{lstlisting}[caption={Basic TCP Server Structure in C}]
#include <stdio.h>
#include <stdlib.h>
#include <string.h>
#include <sys/socket.h>
#include <netinet/in.h>
#include <unistd.h>

#define PORT 8080
#define BACKLOG 10

int main() {
    int server_fd, new_socket;
    struct sockaddr_in address;
    int addrlen = sizeof(address);

    // 1. Create socket file descriptor
    if ((server_fd = socket(AF_INET, SOCK_STREAM, 0)) == 0) {
        perror("socket failed");
        exit(EXIT_FAILURE);
    }

    address.sin_family = AF_INET;
    address.sin_addr.s_addr = INADDR_ANY;
    address.sin_port = htons( PORT );

    // 2. Bind the socket to the specified port
    if (bind(server_fd, (struct sockaddr *)&address, addrlen) < 0) {
        perror("bind failed");
        exit(EXIT_FAILURE);
    }

    // 3. Start listening for incoming connections
    if (listen(server_fd, BACKLOG) < 0) {
        perror("listen");
        exit(EXIT_FAILURE);
    }

    // 4. Accept an incoming connection (blocking call)
    printf("Server listening on port %d...\n", PORT);
    if ((new_socket = accept(server_fd, (struct sockaddr *)&address, 
                           (socklen_t*)&addrlen)) < 0) {
        perror("accept");
        exit(EXIT_FAILURE);
    }

    // 5. Connection successful - Communication happens here (send/recv)
    char *hello = "Hello from server!";
    send(new_socket, hello, strlen(hello), 0);
    printf("Hello message sent.\n");

    // 6. Close the connection
    close(new_socket);
    close(server_fd);

    return 0;
}
\end{lstlisting}

\end{document}